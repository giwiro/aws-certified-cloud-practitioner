\section{Deployments and managing infrastructure at scale}\label{sec:deployments-and-managing-infrastructure-at-scale}

\subsection{Cloudformation}\label{subsec:cloudformation}
\textbf{Infrastructure as code} (json or yaml) that directly represents the resources you want to add, supports almost all AWS services.

\subsection{CDK}\label{subsec:cdk}
Cloud Development Kit (CDK) is an \textbf{Infrastructure as Code} (js/ts, python, java, .net) that is generates cloudformation json/yaml, and deploys it.

\subsection{Elastic BeanStalk}\label{subsec:beanstalk}
\textbf{Platform as a Service} (PaaS) [Similar to \href{https://www.heroku.com}{Heroku}] managed service, it handles: Instance \& OS configuration.
It supports many platforms (go, java, java + tomcat, .net, nodejs, php, ruby, docker, etc).

\subsection{CodeDeploy}\label{subsec:codedeploy}
\textbf{Deploys and upgrades applications} onto servers.
It uses a codedeploy-agent.

\subsection{CodeCommit}\label{subsec:codecommit}
\textbf{Source-control} git-based repository.

\subsection{CodeBuild}\label{subsec:codebuild}
\textbf{Compiles source code, run tests and product artifacts} to be deployed.

\subsection{CodePipeline}\label{subsec:codepipeline}
\textbf{Orchestrate} the different steps to publish code to production (CI/CD).\newline
Code \rightarrow Build \rightarrow Test \rightarrow Provision \rightarrow Deploy

\subsection{CodeArtifact}\label{subsec:codeartifact}
\textbf{Storage for artifacts} (software dependencies)\@.
Works with common dependencies manager (npm, yarn, maven, twine, pip, etc)\@.

\subsection{CodeStar}\label{subsec:codestar}
\textbf{Unified view for developers to do CI/CD} and code in one place.
It automatically creates and manages CodeCommit + CodeBuild + CodeDeploy + CodePipeline + EC2 + Beanstalk.

\subsection{Cloud9}\label{subsec:cloud9}
\textbf{Cloud IDE} in the web browser.
It allows pair programming (real-time).

\subsection{SSM}\label{subsec:ssm}
AWS System Manager (SSM) \textbf{patch, configure and run commands at scale} in EC2 \& On-Premise.
It also \textbf{gets operational insight} of the infrastructure.
It uses ssm-agent.

\subsubsection{SSM Session Manager}
Allows to start a secure shell on your EC2 and on-premise servers.

\subsection{OpWorks}\label{subsec:opworks}
\textbf{Managed \href{https://www.chef.io/}{Chef} and \href{https://www.puppet.com/}{Puppet}}.
It allows you performs resource provisioning and ongoing future server configurations.
