\section{S3}\label{sec:s3}

\subsection{What is}\label{subsec:what-is-s3}
One of the building blocks of AWS that is advertised as \textbf{infinitely scaling} storage.

\subsection{Anatomy}\label{subsec:anatomy}
S3 allows people to store \textbf{objects} ("files") in \textbf{buckets}\, which must have a \textbf{global unique name} but they are \textbf{defined at region level}\.
All the objects have a \textbf{key} that is the full path to the object, e.g. \textit{s3://my-bucket/my\_file.txt}\, and have a \textbf{max size of 5TB}\.

\subsection{Security}\label{subsec:security}

\subsubsection{User based:} IAM policies \(attached to the user or group that will have access\)
\subsubsection{Resource Based:}
\begin{itemize}
	\item \textbf{Bucket Policies:} JSON based rules \(Same as IAM Policy - allow complex rules\)
	\item \textbf{Object Access Control List (ACL):} Simple rules (ALLOW/DENY) over single object, finer-grain control.
	\item \textbf{Bucket Access Control List (ACL):} Simple rules over the whole bucket.
\end{itemize}
\subsubsection{Encryption}

\subsection{Websites}\label{subsec:websites}
S3 can host static websites.
The website URL will be: <bucket-name>.s3-website-<aws-region>.amazonaws.com

\subsection{Versioning}\label{subsec:versioning}
The S3 versioning is enabled at \textbf{bucket level} and it keeps versions of all the objects in S3 when they are changed.
Why use versioning?
\begin{itemize}
	\item{Protect against unintended deletes}
	\item{Easy roll back}
\end{itemize}

\subsection{Access Log}\label{subsec:access-log}
Ability to \textbf{log all access} to S3 buckets (maybe for auditing purposes), e.g\. any request made to S3, authorized or denied.

\subsection{Replication: CRR \&SRR}\label{subsec:replication-crr-srr}
Allow to duplicate in real time an S3 bucket.
In order to achieve that you \textbf{must enable versioning} in source and destination.
\subsubsection{CRR use cases}
\begin{itemize}
	\item{Compliance (high available buckets in case of disaster}
	\item{Replication across accounts}
	\item{Low latency access}
\end{itemize}
\subsubsection{SRR use cases}
\begin{itemize}
	\item{Aggregate log}
	\item{Live replication from production to test account}
\end{itemize}

\subsection{Storage Classes}\label{subsec:storage-classes}
S3 has \textbf{High durability (99.999999999\%)} for all storage classes.

\begin{itemize}
	\item{\textbf{Standard - General Purpose}: 99.99\% available.}
	\item{\textbf{Standard - Infrequent Access (IA)}: 99.9\% available, lower cost but has retrieval fee.}
	\item{\textbf{One Zone - Infrequent Access}: 99.5\% available, lower cost than IA but it is located in a single AZ and has retrieval fee.}
	\item{\textbf{Intelligent Tiering}: 99.9\% available, cost optimized by moving between FA and IA.}
	\item{\textbf{Glacier}: 99.9\% available, data is retained for longer terms (year) and has retrieval fee. Retrieval time:}
	\begin{itemize}
		\item{Expedited (1 to 5 minutes).}
		\item{Standard (3 to 5 hours).}
		\item{Bulk (5 to 12 hours).}
	\end{itemize}
	\item{\textbf{Glacier Deep Archive}: 99.9\% available, data is retained for longer terms (year) and has retrieval fee. Retrieval time:}
	\begin{itemize}
		\item{Standard (12 hours).}
		\item{Bulk (48 hours).}
	\end{itemize}
\end{itemize}

\subsection{Shared Responsibility Model}\label{subsec:shared-responsibility-model}
\begin{itemize}
	\item{S3 Versioning.}
	\item{S3 Bucket Policies.}
	\item{S3 Replica setup.}
	\item{Loggin and Monitoring.}
	\item{S3 Storage classes.}
	\item{Data encryption at rest and in transit.}
\end{itemize}

\subsection{Snowball, Snowball Edge, Snowmobile}\label{subsec:snowball-snowball-edge-snowmobile}
All three services use physical data transportation that helps to move great amount of data in or out of AWS. You pay per data transfer job.

\subsubsection{Snowball:} Can be used to large data cloud migration, disaster recovery.
\subsubsection{Snowball Edge:} 100TB capacity and has computational capability (can process data on the go).
It supports EC2 AMI and Lambda functions.
\subsubsection{Snowmobile:} Transfer exabytes of data.
Best solution if you're going to transfer more than 10PB\.

\subsection{Storage Gateway}\label{subsec:storage-gateway}
S3 can be used as a \textit{"hybrid cloud"} set up.
This means part of you infrastructure is on-premise and part of it is on the cloud.
This allows the on-premise storage to seamlessly use the AWS Cloud.
