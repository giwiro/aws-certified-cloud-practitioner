\section{Other Compute Services}\label{sec:other-compute-services}

\subsection{EC2}\label{subsec:ec2}
(Previously discussed)

\subsection{ECS}\label{subsec:ecs}
Elastic Container Service (ECS) enables the \textbf{launch of Docker containers} but \textbf{you must provision \& maintain the infrastructure (EC2 instances)}\.
AWS takes care of starting and stopping containers and has integration with \textbf{ELB (Application)}\.

\subsection{Fargate}\label{subsec:fargate}
Fargate enables the \textbf{launch of Docker containers} and \textbf{you do not provision the infrastructure (serverless)}\, it is much simpler than ECS\.

\subsection{ECR}\label{subsec:ecr}
Elastic Container Registry (ECR) is a \textbf{private docker registry} on AWS\.

\subsection{AWS Lambda}\label{subsec:aws-lambda}
Lambda is a \textbf{serverless} service which runs \textbf{virtual functions} that are limited by time (execution time), runs \textbf{on demand} (\textbf{event-driven}: gets invoked when needed) and can \textbf{scale automatically}\.
The pricing is based on: number of requests, RAM, execution time.

\subsection{AWS Batch}\label{subsec:aws-batch}
\textbf{Fully managed batch processing} at any scale.
A batch job will start and end.
It will \textbf{dynamically launch EC2 instances (can be Spot Instances)} and the actual batch job will be defined as \textbf{Docker images} and \textbf{run on ECS}\.

\subsection{AWS LightSail}\label{subsec:aws-lightsail}
Standalone service which \textbf{provide fully managed virtual servers, storage, databases and networking}\.
It has low cost, \textbf{predictable pricing} and is an alternative instead of using common AWS services (EC2, RDS, ELB, etc).
That's why it is great for people with \textbf{little cloud experience}.
It has \textbf{high availability} but \textbf{NO AUTO-SCALING and limited AWS integration}.